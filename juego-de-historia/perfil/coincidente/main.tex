\documentclass{article}
\usepackage[spanish]{babel}
\usepackage[utf8]{inputenc}

\usepackage{multicol}
\usepackage{mathtools}
\usepackage{amssymb}
\usepackage{amsmath}
\usepackage{amsfonts}
\usepackage{graphics}
\usepackage{listings}
\usepackage{listingsutf8}
\usepackage{setspace}
\usepackage{xcolor}
\lstset{inputencoding=utf8/latin1}

%\usepackage[margin=1cm]{geometry}
\setlength{\parskip}{0.5cm plus4mm minus3mm}
%\setlength{\parindent}{0pt}

%\renewcommand*{\familydefault}{\sfdefault}


\begin{document}
\title{Integración de la jugabilidad de un videojuego al aprendizaje de historia}

\maketitle

\begin{section}{Problema}
  ¿Se puede integrarse la jugabilidad de un videojuego de acción al aprendizaje de historia?
\end{section}



\begin{section}{Objetivo General}
  Crear e implementar un videojuego educativo de acción que integre a su jugabilidad el aprendizaje de la historia de la guerra del pacífico.
\end{section}


\input{../alcances.tex}
\end{document}
