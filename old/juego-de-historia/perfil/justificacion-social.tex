\begin{subsection}{Social}
  \paragraph{Videojuegos educativos}
  Pese a que el internet ha cambiado la forma en la que los nativos digitales aprenden, el sistema educativo en las escuelas públi- cas de Bolivia, no ha cambiado mucho. Como país, necesitamos adaptarnos a este cambio, y ofrecer una formación educativa de calidad para todos: para los bolivianos principalmente, y para el mundo entero en la medida de lo posible. 

  Algunas ramas de la ciencia, son lo suficientemente generales como para adaptarse a cualquier estudiante, por ejemplo, un estudiante boliviano interesa- do en matemática, podría seguir un curso de matemática del OpenCourseWare del MIT, sin tener ningún otro obstáculo del que tendría un estudiante francés, a excepción, tal vez, del lenguaje. 

  Pero la historia no es una de estas ramas; cada país tiene su propia riqueza cultural e histórica, la historia de cada país, muchas veces, es conocida solamente por los historiadores de ese país, por eso es importante que cada país difunda su propia historia. 

  Por eso es importante liberar la información sobre la historia de Bolivia, y difundirla, no solo entre los bolivianos, si no que a nivel mundial. Para que el mundo entero conozca nuestra historia, necesitamos, al menos, estas dos cosas:

  \begin{itemize}
  \item Hacerlo mediante medios a los que gran parte del mundo pueda acceder. 
  \item Hacer del proceso de aprendizaje algo ameno y divertido.    
  \end{itemize}

  \paragraph{Demanda marítima}
  La historia sobre la guerra del pacífico atrajo la aten- ción de todo el mundo, a finales del siglo XVIII, pero toda esta atención inter- nacional se ha ido perdiendo con el paso del tiempo. Desde entonces, Bolivia es uno de los dos países de América que no cuentan con salida al mar. El 24 de abril de 2013, el gobierno de Bolivia inició formalmente una demanda marítima, sin embargo, la demanda no es internacionalmente conocida a nivel popular.
\end{subsection}
