\begin{section}{Problema}
  Pese a que existen muchos juegos sobre historia, no se pudo encontrar ninguno que no utilice la historia solo como ``fondo'': la historia, en todos los video juegos del marco teórico de la presente investigación (a excepción de algunos que entran en la categoría de edutainment), no utilizan a la historia como piedra angular de su jugabilidad, y por lo tanto, existe el riesgo de que el jugador se distraiga con otros detalles de la jugabilidad, y no preste atención a los detalles importantes de la historia que le están contando. 

  No existe videojuego que, mediante su jugabilidad, motive al jugador a prestar atención a los detalles correctos, de una forma divertida y basada en multimedia no-textual, sin recurrir al abuso de lectura de textos largos, o al abuso de cuestionarios de respuesta múltiple.

  El jugador podría ganar el juego sin que sea necesario que entienda la historia que está sucediendo, únicamente siguiendo las reglas del juego, explotando únicamente sus habilidades motrices y estratégicas.

  \textbf{¿Será que un juego educativo de historia que implemente prestarle atención a los detalles correctos, a una jugabilidad con acción y agilidad, causará que los jugadores no puedan ganar el juego sin haber aprendido historia y se sientan motivados de hacerlo?}
\end{section}
